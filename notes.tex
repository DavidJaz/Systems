\section{Introduction}


\section{Generalized Lenses}

In this section, we will recall Spivak's definition of a \emph{generalized lens}.

The basic ingredient in the definition of a generalized lens will be a category
$\Ca$ of \emph{contexts}, and for each context $C \in \Ca$, a category $\Aa(C)$
of \emph{actions} available in the context $C$. The category of actions must
vary with the context contravariantly: if $f : C' \to C$ is any change of
context, we must have a \emph{pullback} functor $\Aa(f) : \Aa(C) \to \Aa(C')$
that recontextualizes actions along $f$.

Such a contravariant (pseduo-)functor $\Aa : \Ca\op \to \Cat$ is called an
\emph{indexed category}, indexed by the category $\Ca$.

\begin{defn}
Let $\Aa : \Ca\op \to \textbf{Cat}$ be an indexed category
assigning each \emph{context} $C \in \Ca$ to its category of \emph{actions}
$\Aa(C)$.

The category $\textbf{Lens}_{\Aa}$ of $\Aa$-lenses has:
\begin{itemize}
\item Objects pairs $\lens{A}{C}$ of a context $C \in \Ca$ and an action $A \in
  \Aa(C)$ in the context of $C$.
\item Morphisms (called \emph{lenses}) $\lens{f^{\sharp}}{f} : \lens{A}{C} \to
  \lens{A'}{C'}$ are pairs consisting of a recontexualization $f : C \to C'$ and
  a morphism $f^{\sharp} : \Aa(f)(A') \to A$ of actions.
\end{itemize}
Composition of $\lens{f^{\sharp}}{f} : \lens{A}{C} \to \lens{A'}{C'}$ with
$\lens{g^{\sharp}}{g} : \lens{A'}{C'} \to \lens{A''}{C''}$ is given by
$$\lens{g^{\sharp}}{g} \circ \lens{f^{\sharp}}{f} :\equiv \lens{f^{\sharp} \circ
\Aa(f)(g^{\sharp})}{g \circ f}.$$
\end{defn}

Suppose furthermore that $\Aa: \Ca\op \to \Cat$ is a \emph{monoidal} indexed
category -- a weak monoid in the 2-category of indexed categories. This means
that $\Ca$ is a monoidal category with tensor $\otimes$ and
unit $\One$, and that $\Aa : \Ca\op \to \Ca$ is a lax monoidal functor, meaning
that it is equipped with functors
$$- \boxtimes - : \Aa(C) \times \Aa(C') \to \Aa(C \otimes C') \quad\quad
\One_{\boxtimes} : \ast \to \Aa(\One)$$
which are natural in $C$ and $C'$ and satisfy the evident laws. Then by the work
of [christina], the Grothendieck construction of a monoidal indexed category
admits a canonical monoidal structure.

\begin{defn}
Let $\Aa : \Ca\op \to \Cat$ be a monoidal indexed category. Then the category
$\textbf{Lens}_{\Aa}$ of $\Aa$-lenses admits a monoidal structure given by
$$\lens{A}{O} \otimes \lens{A'}{O'} = \lens{A \boxtimes A'}{O \otimes O'}$$
and similarly for lenses.
\end{defn}


Let's see a number of examples of this definition.
\subsection{Example: Lenses in a Cartesian Category}
Let $\Ca$ be a cartesian category --- every pair of objects $A$ and $B$ admits a
cartesian product $A \times B$, and there is a terminal object $\ast \in \Ca$.
Every object $C \in \Ca$ is therefore a comonoid with
$$\varepsilon : C \to \ast \quad\quad\quad\quad \delta : C \to C \times C$$
given by the terminal morphism $\varepsilon$ and diagonal morphism $\delta$. The
functor $X \mapsto C \times X : \Ca \to \Ca$ therefore inherits the natural
structure of a comonoid under composition: it is a comonad with
$$\varepsilon_C : C \times X \to X \quad\quad\quad\quad \delta_C : C \times X \to C
\times C \times X$$
counit $\varepsilon_C := \pi_2$ the second projection and comultiplication
$\delta_C := \delta \times X$ given by the diagonal in the left component. We
recall the definition of the \emph{CoKleisli category} $\textbf{CoKleisli}(C
\times -)$ for this comonad.
\begin{defn}
The \emph{CoKleisli category} $\textbf{CoKleisli}(C \times -)$ for the comonad
$C \times -$ has:
\begin{itemize}
\item Objects the objects of $\Ca$.
  \item Morphisms $f : X \to Y$ in $\textbf{CoKleisli}(C \times -)$ are maps $f : C \times X \to Y$ in $\Ca$.
\end{itemize}

CoKleisli composition of $f : C
\times X \to Y$ with $g : C \times Y \to Z$ is given by
$$C \times X \xto{\delta_C} C \times C \times X \xto{C \times f} C \times Y
\xto{g} Z$$
or $g \circ f :\equiv g \circ (C \times f) \circ \delta_C$. The identity is the
second projection $\varepsilon_C : C \times X \to X$.
\end{defn}
[something about cokleisli maps being maps in the context of $C$]

Given any function $f : C \to C'$, we get a functor $\textbf{CoKleisli}(f \times
-) : \textbf{CoKleisli}(C' \times -) \to \textbf{CoKleisli}(C \times -)$ sending
$X$ to $X$ and sending $g : C' \times X \to Y$ to
$$C \times X \xto{f \times X} C'\times X \xto{g} Y$$
or $g \circ (f \times X)$. This gives us a contravariant functor
$$\textbf{CoKleisli}(+ \times -) : \Ca^{op} \to \Cat$$
sending each $C \in \Ca$ to the CoKleisli category $\textbf{CoKleisli}(C \times -)$.

\begin{defn}
A \emph{lens} in the cartesian category $\Ca$ is a $\textbf{CoKleisli}(+ \times
-)$-lens. The category $\textbf{Lens}_{\Ca}$ of lenses in $\Ca$ therefore has:
\begin{itemize}
\item Objects pairs $\lens{A}{C}$ with $C \in \Ca$ and $A \in
  \textbf{CoKleisli}(C \times -)$ (and therefore also an object of $\Ca$).
\item Morphisms \emph{lenses} $\lens{f^{\sharp}}{f} : \lens{A}{C} \to
  \lens{A'}{C'}$ with $f : C \to C'$ and $f^{\sharp} : C \times A' \to
  A$.
\end{itemize}

Composition of lenses is given by the formula
$$(g \circ f)^{\sharp}(c, a'') :\equiv f^{\sharp}(c, g^{\sharp}(f(c), a'')).$$
\end{defn}

Now, $\Ca$ is a monoidal category with tensor $\times$ and
unit $\ast$. Furthermore, we may define
$$- \boxtimes - : \textbf{CoKleisli}(C \times -) \to \textbf{CoKleisli}(C'
\times -) \to \textbf{CoKleisli}((C \times C') \times -)$$
by $A \boxtimes B = A \times B$ and with $f \boxtimes g$ being
$$(C \times C') \times (A \times B) \xto{\sim} (C \times A) \times (C' \times B)
\xto{f \times g} A' \times B'$$
with $f : C \times A \to A'$ and $g : C' \times B \to B'$.

This gives a monoidal struction on $\textbf{Lens}_{\Ca}$, where
$$\lens{A}{O} \otimes \lens{A'}{O'} = \lens{A \times A'}{O \times O'}$$
and
$$\lens{f^{\sharp}}{f} \otimes \lens{g^{\sharp}}{g} = \lens{f \boxtimes g}{f \times g}.$$

\subsection{Example: Prisms and Wiring Diagrams}

A \emph{wiring diagram} is a way of wiring together a bunch of \emph{boxes} into
a larger box:

[picture]

Wiring diagrams may be composed by filling in the boxes and erasing their
boundaries:

[picture]

A \emph{box} has a set of input wires (drawn on the left) and output wires
(drawn on the right):

[picture]

We can place boxes together by taking the disjoint union of their input and
output wires respectively.

A wiring diagram of a set of inner boxes into an outer box follows these simple rules:
\begin{itemize}
\item Every output of the outer box is wired to exactly one output of the inner
  boxes.
  \item Every input of the inner boxes is wired either to an input of the outer
    box, or an output of an inner box.
\end{itemize}

If we write $\lens{I_{\text{in}}}{O_{\text{in}}}$ for inner boxes with input set
$I_{\text{in}}$ and output set $O_{\text{in}}$ and
$\lens{I_{\text{out}}}{O_{\text{out}}}$ for the outer box, then the rules for a
wiring diagram give us a pair of functions:
\begin{itemize}
\item An assignment $f : O_{\text{out}} \to O_{\text{in}}$ assigning each outer
  output to an inner output.
\item An assignment $f^{\sharp} : I_{\text{in}} \to O_{\text{in}} +
  I_{\text{out}}$ assigning each inner input either to an inner output or an
  outer input.
\end{itemize}

This is dual to the notion of a lens.

A \emph{prism} is the categorical dual of a lens. If $\Ca$ is a cocartesian
category, having all finite coproducts, then every object $C \in \Ca$ is a
monoid with regards to the coproduct:
$$\eta : \emptyset \to C \quad\quad\quad\quad \mu : C + C \to C$$
are the initial morphism and the codiagonal respectively. This means that the
functor $X \mapsto C + X$ is a monad and we can form its Kleisli category
$\textbf{Kleisli}(C + -)$. This gives us a \emph{covariant} functor
$\textbf{Kleisli}((-) + (-))\op : \Ca \to \Ca$ sending each $C$ to the
\emph{opposite} of its Kleisli category, which we will interpret as a
\emph{contravariant} functor on $\Ca\op$. 

\begin{defn}
A \emph{prism} in a cocartesian category $\Ca$ is a lens in $\Ca\op$. As a
generalized lens, it is a $\textbf{Kliesli}((-) + (-))\op$-lens.

A \emph{wiring diagram} is a prism in the category of finite sets.
\end{defn}

\subsection{Example: $M$-Lenses for a Strong Monad}

Let $\Ca$ be a cartesian category. A \emph{strong monad} $M : \Ca \to \Ca$ is a
monad equipped with a \emph{strength} $\sigma : C \times MX \to M(C \times X)$,
natural in $C$ and $X$ and satisfying a few natural laws (see [somewhere]):

  Every monad on the category of sets is strong, as is every monad in Haskell.
  The strength can be written in Haskell as:
\begin{verbatim}
strength :: Monad m => c -> m x -> m (c, x)
strength c mx = do
  x <- mx
  return (c, x)
\end{verbatim}

  Despite this simple definition, the strength is a useful concept for a number
  of monads. For example, if $M : \Set \to \Set$ is the \emph{probability
    distribution monad} sending a set $X$ to the set of (finitely supported)
  probability distributions on it, then the strength $\sigma : C \times MX \to
  M(C \times X)$ sends $(c, p)$ to the distribution $\delta_c p$ which assigns
  probability $p_x$ to $(c', x)$ if $c' = c$ and $0$ otherwise.
  
  There is another way to describe a strong monad on a cartesian category.
\begin{prop}
Let $M : \Ca \to \Ca$ be a monad on a cartesian category $\Ca$. The data of a
strength for $M$ is equivalent to the data of a natural assignment of a
distributive law $\sigma : (C \times -) \circ M \to M \circ (C \times -)$ of the
comonad $C \times -$ over $M$ to every object $C \in \Ca$.
\end{prop}

  Therefore, we can define a \emph{BiKleisli} category $\textbf{BiKleisli}(C
  \times -, M)$ as follows:
\begin{defn}
Let $M : \Ca \to \Ca$ be a strong monad on a cartesian category $\Ca$ and let $C
\in \Ca$. Then the \emph{BiKleisli} category $\textbf{BiKleisli}(C \times -, M)$
has:
\begin{itemize}
\item Objects the objects of $\Ca$.
\item Morphisms $X \to Y$ are maps $C \times X \to MY$ in $\Ca$.
\end{itemize}
Composition of $f : C \times X \to MY$ with $g : C \times Y \to MZ$ is given by
$$C \times X \xto{\delta_C} C \times C \times X \xto{C \times f} C \times MY
\xto{\sigma} M(C \times Y) \xto{M(g)} MMZ \xto{\mu} MZ,$$
or $g \circ f :\equiv \mu \circ M(g) \circ \sigma \circ C \times f \circ \delta_C$.
\end{defn}

\begin{defn}
Let $M : \Ca \to \Ca$ be a strong monad on a cartesian category. An
\emph{$M$-lens} is a lens for the functor $\textbf{BiKleisli}((-) \times (-), M)
: \Ca\op \to \Ca$.
\end{defn}
[this corresponds to the notion of monadic lens from ``reflections on monadic
lenses'' by abou-saleh cheney gibbons mckinna and stevens]

We can define the $M$-lens composite $\lens{g^{\sharp}}{g} \circ \lens{f^{\sharp}}{f}$ in Haskell as:
\begin{verbatim}
lensComposite :: Monad m => (c  -> c')
                         -> (c' -> c'')
                         -> ((c, a') -> m a) 
                         -> ((c', a'') -> m a') 
                         -> (c, a'') -> m a
lensComposite f g f# g# c' a'' = do
  a' <- g# (f c, a'')
  f# (c, a')
\end{verbatim}

Given a strength $\sigma : C \times MX \to M(C \times X)$, we can define a
``mix'' morphism $m : MX \times M Y \to M(X \times Y)$ to be the composite
$$MX \times MY \xto{\sigma} M(X \times MY) \xto{\sim} M(MY \times X)
\xto{M\sigma} MM(X \times Y) \xto{\mu} M(X \times Y).$$

Such a morphism is an equivalent way to define a strong monad. The mix function
may be defined in Haskell as:
\begin{verbatim}
mix :: Monad m => m x -> m y -> m (x, y)
mix mx my = do
  x <- mx
  y <- my
  return (x, y)
\end{verbatim}
For the probability monad, the mix function sends a probability distribution $p$
on $X$ and a probability distribution $q$ on $Y$ to their joint distribution
$pq$ on $X \times Y$. For the list monad, the mix function zips together two
lists. 

Using the mix morphism $m : MX \times MY \to M(X \times Y)$, we can give the
structure of a monoidal indexed category on $\textbf{BiKleisli}((-) \times (-),
M)$. Define $A \boxtimes B$ to be $A \times B$ and $f \boxtimes g$ to be the
composite
$$(C \times C') \times (A \times B) \xto{\sim} (C \times A) \times (C' \times B)
\xto{f \times g} M A' \times M B' \xto{m} M(A' \times B'),$$
where $f : C \times A \to M A'$ and $g : C' \times B \to M B'$. 


\subsection{Example: Dependent Lenses in a Category with Display Maps}

\begin{defn}
A \emph{category with display maps} is a category $\Ca$ with a class $D$ of its
morphisms such that for any $d : X \to Y$ in $D$ and $f : Z \to Y$, the pullback
$f^{\ast}d : Z \times_Y Y$ exists and is in $D$.
\end{defn}

\begin{ex}
Let $\textbf{Mfd}$ denote the category of smooth manifolds and smooth functions.
Then the class $\textbf{Subm}$ of submersions is a class of display maps.
\end{ex}

\begin{defn}
A \emph{dependent lens} in a category $(\Ca, D)$ with display maps is a $(\Ca
\downarrow_{D} -)$-lens, where $\Ca \downarrow_D - : \Ca^{\op} \to \textbf{Cat}$ sends $C
\in \Ca$ to the full subcategory of the slice category $\Ca \downarrow -$
spanned by the maps in $D$.
\end{defn}

\begin{rmk}
The motiviation for the name ``dependent lens'' comes from dependent type
theory. A display map $f : A \to B$ can be through of as a typing judgement $b : B
\vdash A_b : \Type$. With this interpretation, a dependent lens $\lens{A}{C} \to
\lens{A'}{C'}$ consists of a function $f : C \to C'$ and a dependent function 
$$f^{\sharp} : \dprod{c : C} A'_{f(c)} \to A_c.$$

\end{rmk}

If we further ask that the product $f \times g$ of two display maps $f,\, g \in
D$ is also a display map, then $(f, g) \mapsto f \times g$ gives a monoidal
structure on the indexed category $\Ca \downarrow_D -$. Since the product of
submersions is a submersion, this works for our example above.
\subsection{Generalized Lens Categories via the Grothendieck Construction}

To those who have seen the Grothendieck construction before, the above
definition of generalized lens categories must feel familiar, if a bit off. In
fact, the category of generlized lenses $\textbf{Lens}_{\Aa}$ \emph{is} a
Grothendieck construction, but applied to $\Aa$'s pointwise opposite $\Aa\op$.
\begin{prop}
Let $\Aa : \Ca\op \to \textbf{Cat}$ be a contravariant (pseudo-)functor. Let
$\Aa\op : \Ca\op \to \textbf{Cat}$ denote the composite $\Ca\op \xto{\Aa}
\textbf{Cat} \xto{(-)\op} \textbf{Cat}$, the pointwise opposite. Then
$$\textbf{Lens}_{\Aa} = \int_{C : \Ca} \Aa(C)\op$$
where $\int_{C : \Ca} \Aa(C)\op$ is the (contravariant) Grothendieck construction applied to
$\Aa\op$.
\end{prop}

As a corollary, we can import a few results from the theory of Grothendieck
fibrations.
\begin{thm}{Recognition of Limits in Lens Categories}\label{thm:limits.in.lens.cats}
Let $D : J \to \textbf{Lens}_{\Aa}$ be a diagram in a generalized lens category.
If $U \circ D : J \to \Ca$ admits a limit $L$ and the pulled back diagram
$D^{\ast} : J \to \Aa(L)$ admits a colimit $A$, then $\lens{A}{L}$ is the limit
of $D$ in $\textbf{Lens}_{\Aa}$.
\end{thm}
\begin{proof}

\end{proof}

\begin{cor}
Suppose that $\Ca$ has $J$-indexed limits and that for each $C \in \Ca$,
$\Aa(C)$ has $J$-indexed colimits and $\Aa(f)$ preserves them for $f : C \to
C'$. Then $\textbf{Lens}_{\Aa}$ has $J$-shaped limits and the forgetful functor
$\textbf{Lens}_{\Aa} \to \Ca$ preserves them.
\end{cor}
\begin{proof}

\end{proof}

As a corollary, we find that pullbacks of lenses can be constructed in $\Ca$.
\begin{cor}
Let \[J = \begin{tikzcd} & \bullet \arrow[d] \\ \bullet \arrow[r] &
  \bullet \end{tikzcd}\]
so that a limit of a diagram $D : J \to \textbf{Lens}_{\Aa}$ is a pullback. Such
a diagram admits a limit if and only if the underlying diagram $U \circ D : J
\to \Ca$ admits a limit. 
\end{cor}
\begin{proof}
Note that since $J$ has a terminal object, every category has $J$-shaped colimits and every functor
preserves them (the colimit is just the tip of the cospan). Therefore, the
conditions of Theorem \ref{thm:limits.in.lens.cats} are satisfied.
\end{proof}

\subsection{Functoriality of the Generlized Lens Construction}

Since the generalized lens construction is just an instance of the Grothendieck
construction, we get its functoriality for free.

\begin{prop}
  Let $(F, \phi) : \Aa \to \Ba$ be a functor of indexed categories. That is,
  \[
    \begin{tikzcd}
      \Ca \arrow[rd,"\Aa"{name=U} ] \arrow[dd,"F"']& \\
      & \textbf{Cat} \\
      \Da \arrow[Rightarrow,
      from=U, shorten <= 1em, shorten >= 1em, "\phi"] \arrow[ur, "\Ba"'] 
    \end{tikzcd}
  \]
  $F : \Ca \to \Da$ is a functor and $\phi : \Aa \to \Ba \circ F$ is a natural
  transformation. Then we have an induced functor
  $$(F, \phi)_{\ast} : \textbf{Lens}_{\Aa} \to \textbf{Lens}_{\Ba}$$
  given by sending $\lens{A}{C}$ to $\lens{\phi A}{F C}$ and
  $\lens{f^{\sharp}}{f}$ to $\lens{\phi f^{\sharp}}{F f}$.

  If $\Aa$ and $\Ba$ are monoidal indexed categories and $(F, \phi)$ is a lax
  monoidal indexed functor (see [christina]), then the induced functor $(F,
  \phi)_{\ast}$ will also be lax monoidal.
\end{prop}

\begin{ex}
Suppose that $\alpha : M \to N$ is a strong monad morphism between strong monads $M$
and $N$. This means that, in addition to $\alpha$ being a monad morphism, the
following diagram commutes:
\[
  \begin{tikzcd}
    C \times M X \arrow[r,"\sigma_M"] \arrow[d, "C \times \alpha"'] & M(C \times X) \arrow[d,"\alpha"] \\
    C \times N X \arrow[r,"\sigma_N"'] & N(C \times X)
  \end{tikzcd}
\]

Then we get an indexed
functor $(\id, \alpha) : \textbf{BiKleisli}((-) \times (-), M) \to
\textbf{BiKleisli}((-) \times (-), N)$ given by sending $f : C \times X \to M Y$
to $\alpha_Y \circ f : C \times X \to N Y$.

This induces a functor $\textbf{Lens}_{M} \to \textbf{Lens}_{N}$ which sends an
$M$-lens $\lens{f^{\sharp}}{f}$ to the $N$-lens $\lens{\alpha \circ f^{\sharp}}{f}$.

Of particular importance is the unique monad morphism $\eta : \id \to M$ given
by the unit of the monad $M$. That this monad morphism is strong is one of the
axioms of a strength for a monad. This means that we have a functor
$$\textbf{Lens}_{\Ca} \to \textbf{Lens}_M$$
for any strong monad $M$ on $\Ca$, lifting any lens into an $M$-lens.
\end{ex}


\begin{ex}
Suppose that $\Ca$ is a cartesian category. Let $\textbf{Fin} \downarrow \Ca$ be
the category of $\Ca$-labeled finite sets. 

There is a contravariant functor $\prod : \textbf{Fin} \downarrow \Ca \to
\Ca\op$ taking a $\Ca$-labeled finite set $\ell : X \to \Ca$ to the product
$\prod_{x \in X} \ell_x$.

The functor $\prod$ sends coproducts to products, so given a $\Ca$-labeled finite set $\ell : X \to \Ca$, we get a functor
$\prod : \textbf{Kleisli}(\ell + -)\op \to \textbf{CoKleisli}(\prod \ell \times
-)$. This gives us a functor of indexed categories.

Finally, by the functoriality of the generalized lens construction, we get a
functor
$$\textbf{WD}_{\Ca} \to \textbf{Lens}_{\Ca}$$
interpreting any wiring diagram as a lens.
\end{ex}

\section{Sections}

\begin{defn}
Let $\Aa : \Ca\op \to \textbf{Cat}$. A \emph{section} $\varepsilon$ of $\Aa$ is an
assignment of an object $\varepsilon C \in \Aa(C)$ to each $C \in \Ca$ and
$\varepsilon_f : \varepsilon C \to \Aa(f)(C')$ for each $f : C \to C'$ such that
\begin{itemize}
\item $\varepsilon_{\id} = \id$.
\item $\varepsilon_{g \circ f} = \Aa(g)(\varepsilon_f) \circ \varepsilon_g$.
\end{itemize}

If $\Aa$ is a monoidal indexed category, then we say that $\varepsilon$ is a
\emph{monoidal} section if it is equipped with a morphism $\zeta : \varepsilon
\One_{\otimes} \to \One_{\boxtimes}$ and morphisms $\ell : \varepsilon(C \otimes
C') \to \varepsilon C'
\boxtimes \varepsilon C'$ which commute with the
structure maps of $\boxtimes$ and $\otimes$, and which are natural in $C$ and
$C'$ in the sense that the following diagram commutes: 
\[
  \begin{tikzcd}
   \varepsilon(C \otimes C')  \arrow[r,"\ell" ] \arrow[d, "\varepsilon_{f \otimes g}"] & \varepsilon C \boxtimes \varepsilon C'  \arrow[d,"\varepsilon_f \boxtimes
    \varepsilon_g"' ]\\
    \Aa(f \otimes g)( \varepsilon(D \otimes D') ) \arrow[r, "\ell"'] & \Aa(f)( \varepsilon D ) \boxtimes \Aa(g)( \varepsilon D' )
  \end{tikzcd}
\]

That is, $\varepsilon$ is a section of the (contravariant) Grothendieck
construction $\int \Aa \to \Ca$ of $\Aa$.\footnote{Note that this is \emph{not} the
  category of $\Aa$-lenses, which is the contravariant Grothendieck construction
of the pointwise opposite $\Aa\op$.}
 If $\Aa$ is a monoidal, then we ask that
 $\varepsilon$ be an \emph{oplax} monoidal section.\footnote{In our examples,
   our sections will be \emph{pseudo}-monoidal, that is, the maps $\zeta$ and $\ell$
 will be isomorphisms. But we only use the oplax direction abstractly.}
\end{defn}

\begin{itemize}
\item Let $\Ca$ be a cartesian category. The $\varepsilon C := C$ and $\varepsilon_f
  := f \circ \pi_2$ gives a section of $\textbf{CoKleisli}((-) \times (-)) :
  \Ca\op \to \Cat$. This section is monoidal, with both $\zeta : \ast \times \ast \to
  \ast$ and $\ell : (C \times C') \times (C \times C') \to
  C \times C'$ defined to be the identity, $\pi_2$.
\item Similarly, if $M : \Ca \to \Ca$ is a strong monad on a cartesian category,
  then we can set $\varepsilon C := C$ and $\varepsilon_f := \eta \circ f \circ
  \pi_2$ to get a section of $\textbf{BiKleisli}((-) \times (-), M)$. This
  section is also monoidal, with $\zeta$ and $\ell$ both identities $\eta \circ \pi_2$. 
\item Consider the category $(\textbf{Mfd}, \textbf{Subm})$ of smooth manifolds
  equipped with submersions as display maps. The tangent bundle functor sending
  a smooth manifold to the projection $\pi : TC \to C$ of the tangent bundle of
  $C$ gives a section of $\textbf{Mfd} \downarrow_{\textbf{Subm}} -$ by setting
  $\varepsilon_f := (\pi, df) : TC \to C \times_C' TC'$. 

  This section is also monoidal, with $\zeta : T\ast \to \ast$ and $\ell : T(C \times C') \to TC \times TC'$
  the canonical isomorphisms.
\end{itemize}

\subsection{Aside: Simple and Lawful Lenses}

\begin{defn}
Let $\varepsilon$ be a section of $\Aa : \Ca\op \to \textbf{Cat}$. A \emph{simple
  $\Aa$-lens} is a lens of the form
$\lens{f^{\sharp}}{f} : \lens{\varepsilon C}{C} \to \lens{\varepsilon D}{D}$.

Given a simple lens $\lens{f^{\sharp}}{f}$, consider the following \emph{laws}:
\begin{itemize}
  \item $\lens{f^{\sharp}}{f}$ is said to satisfy \emph{getput} if $\varepsilon_f
    \circ f^{\sharp} = \id_{\Aa(f)(\varepsilon D)}$.
  \item $\lens{f^{\sharp}}{f}$ is said to satisfy \emph{strong putget} if
    $f^{\sharp} \circ \varepsilon_f = \id_{\varepsilon C}$.
\end{itemize}
\end{defn}

In the category $\textbf{Lens}_{\Ca}$ of lenses in a cartesian category $\Ca$,
these yield the laws
$$f(f^{\sharp}(c, d)) = d \quad\text{and}\quad f^{\sharp}(c, f(c')) = c'.$$

These are called the \emph{getput} and \emph{strongputget} laws in (cite lens
laws paper).


\section{Open Dynamical Systems}

\begin{defn}
Let $\lens{I}{O}$ be an object of $\textbf{Lens}_{\Aa}$ for an indexed category
$\Aa : \Ca\op \to \Cat$, and that $\varepsilon$ is a section of $\Aa$. An
\emph{$\lens{I}{O}$-dynamical system} consists of:
\begin{itemize}
\item An object $S \in \Ca$ of \emph{states}, and
  \item A lens $\lens{u}{r} : \lens{\varepsilon S}{S} \to \lens{I}{O}$,
    consisting of a \emph{readout} $r : S \to O$ and an \emph{update} $u :
    \Aa(r)(I) \to \varepsilon S$.
  \end{itemize}

A map $\phi : \lens{u}{r} \to \lens{u'}{r'}$ of $\lens{I}{O}$-dynamical systems
is a map $\phi : S \to S'$ satisfying the following laws:
\begin{itemize}
\item Observational equivalence:
  \[
    \begin{tikzcd}[row sep=tiny]
      S \arrow[dd, "\phi"']  \arrow[dr, "r"] & \\
      & O \\
      S' \arrow[ur, "r'"'] &
    \end{tikzcd}
  \]
\item Covariance:
  \[
    \begin{tikzcd}
      \Aa(r)(I) \arrow[r, "u"] \arrow[d, equal] & \varepsilon S \arrow[d, "\varepsilon_{\phi}"] \\
      \Aa(\phi)(\Aa(r')(I)) \arrow[r, "\Aa(\phi)(u')"'] & \Aa(\phi)(\varepsilon S')
    \end{tikzcd}
  \]
\end{itemize}
We will denote by $\textbf{Dyn}_{\lens{I}{O}}$ the category of
$\lens{I}{O}$-dynamical systems. 
\end{defn}

We can see that the composite $\psi \circ \phi$ in $\textbf{Dyn}_{\lens{I}{O}}$ is well defined by
considering the following diagram
\[
  \begin{tikzcd}
      \Aa(r)(I) \arrow[r, "u"] \arrow[d, equal] & \varepsilon S \arrow[d,
      "\varepsilon_{\phi}"] \arrow[dd, bend left = 60, "\varepsilon_{\psi \circ \phi}"] \\
      \Aa(\phi)(\Aa(r')(I)) \arrow[r, "\Aa(\phi)(u')"] \arrow[d, equal] & \Aa(\phi)(\varepsilon
      S') \arrow[d, "\Aa(\phi)(\varepsilon_{\psi})"']\\
      \Aa(\phi)(\Aa(\psi)(r'')(I)) \arrow[r,"\Aa(\phi)(\Aa(\psi)(u''))"' ] &
      \Aa(\phi)(\Aa(\psi)(\varepsilon S''))
  \end{tikzcd}
\]
and recalling that $\Aa(\psi \circ \phi) = \Aa(\phi)\circ \Aa(\psi)$.

Post-composing by a lens $\lens{f^{\sharp}}{f} : \lens{I}{O} \to \lens{I'}{O'}$
gives a functor $\textbf{Dyn}_{\lens{I}{O}} \to \textbf{Dyn}_{\lens{I'}{O'}}$.
This gives us a \emph{covariant} functor
$$\textbf{Dyn}_{(-)} : \textbf{Lens}_{\Aa} \to \Cat.$$
We define the category $\textbf{Dyn}_{(\Aa, \varepsilon)}$ to be the
\emph{covariant} Grothendieck construction of $\textbf{Dyn}_{(-)}$.

Explicitly, a $(\Aa,\varepsilon)$-\emph{dynamical system} consists of:
\begin{itemize}
\item A context $S \in \Ca$ of \emph{states}.
\item A context $O \in \Ca$ of \emph{outputs}.
\item A type of actions $I \in \Aa(O)$ of \emph{inputs}, contextualized by the
  outputs.
\item An $\Aa$-lens $\lens{u}{r} : \lens{\varepsilon S}{S} \to \lens{I}{O}$
  consisting of a \emph{readout} $r : S \to O$ and an \emph{update} $u :
  \Aa(r)(I) \to \varepsilon S$.
\end{itemize}

A \emph{map} $\lens{u}{r} \to \lens{u'}{r'}$ of dynamical systems is a pair $\left(\phi, \lens{f^{\sharp}}{f}\right)$
consisting of a map $\phi : S \to S'$ and a lens $\lens{f^{\sharp}}{f} :
\lens{I}{O} \to \lens{I'}{O'}$ such that the following diagrams commute:
\[
\begin{tikzcd}
S \arrow[r,"r" ] \arrow[d,"\phi"' ]& O \arrow[d,"f" ]\\
S' \arrow[r,"r'"' ] & O'
\end{tikzcd}
\quad\quad
\begin{tikzcd}
  \Aa(r)(\Aa(f)(I')) \arrow[r,"\Aa(r)(f^{\sharp})"] \arrow[d, equal] &\Aa(r)(I) \arrow[r, "u"] &\varepsilon S \arrow[d, "\varepsilon_{\phi}"]
  \\
  \Aa(\phi)(\Aa(r')(I')) \arrow[rr,"\Aa(\phi)(u')"'] & &\Aa(\phi)(\varepsilon S') \\
\end{tikzcd}
\]

If $\Aa : \Ca\op \to \Cat$ is a monoidal indexed category, then
$\textbf{Dyn}_{(-)}$ is also monoidal with $- \boxtimes - :
\textbf{Dyn}_{\lens{I}{O}} \times \textbf{Dyn}_{\lens{I'}{O'}} \to
\textbf{Dyn}_{\lens{I}{O} \otimes \lens{I'}{O'}}$ given by
$$\lens{u}{r} \boxtimes \lens{u'}{r'} := \lens{(u \boxtimes u') \circ \ell}{r
  \otimes r'}$$
and $\One_{\boxtimes} := \lens{\zeta}{\id_{\One_{\otimes}}}$.

\subsection{Example: Open Discrete Dynamical Systems}

For $\Aa(C) = \textbf{CoKleisli}(C \times -)$ and $\varepsilon$ as in Example
[cite], an $(\Aa, \varepsilon)$ dynamical system is as follows:

\begin{defn}
  An open discrete dynamical system in the category of sets consists of:
  \begin{itemize}
    \item A set $S$ of \emph{states}, a set $O$ of \emph{outputs}, and a set $I$
      of \emph{inputs}.
    \item A readout function $r : S \to O$ and an update function $u : S \times
      I \to S$.
  \end{itemize}
  A morphism $\left( \phi, \lens{f^{\sharp}}{f} \right) : \left( S, \lens{I}{O},
  \lens{u}{r} \right) \to \left( S', \lens{I'}{O'}, \lens{u'}{r'} \right)$
consists of a function $\phi : S \to S'$ and a lens $\lens{f^{\sharp}}{f} :
\lens{I}{O} \to \lens{I'}{O'}$ satisfying the following two laws:
\begin{itemize}
\item (Observational equivalence) $r'(\phi(s)) = f(r(s))$ for all $s \in S$.
\item (Covariance) $u'(\phi(s), j) = \phi(u(s, f^{\sharp}(r(s), j)))$ for all $s
  \in S$ and $j \in I'$.
\end{itemize} 
\end{defn}

If $\lens{f^{\sharp}}{f} : \lens{I}{O} \to \lens{I'}{O'}$ is a lens
and $\lens{u}{r} : \lens{S}{S} \to \lens{I}{O}$ is a discrete dynamical system,
then we get a new dynamical system $\lens{u'}{r'} := \lens{f^{\sharp}}{f} \circ
\lens{u}{r}$. In full:
\begin{align*}
u'(s, j) &:= u(s, f^{\sharp}(r(s), j)), \\
r'(s) &:= f(r(s)).
\end{align*}
Then $\left( \id, \lens{f^{\sharp}}{f} \right)$ becomes a morphism from
$\lens{u}{r}$ to $\lens{u'}{r'}$. 


An open discrete dynamical system gives a stream transformation from inputs to
outputs by running the system on a stream of inputs. Given a state
$s \in S$ and a stream of inputs $i \in I^{\omega}$, we get a stream of outputs
$\lens{u}{r}_{\ast}(s, i) \in O^{\omega}$ defined by
\begin{align*}
  \text{head } \lens{u}{r}_{\ast}(s, i) &:= r(s) \\
  \text{tail } \lens{u}{r}_{\ast}(s, i) &:= \lens{u}{r}_{\ast}(u(s, \text{head }i), \text{tail } i)
\end{align*}
or, less formally
$$(r(s), r(u(s, i_0)), r(u(u(s, i_0), i_1)), \ldots)$$

\subsection{Example: Open Markov Systems}

Let $\Ca$ be the category of sets and let $D$ be the probability monad on $\Ca$,
sending each set to the set of (finitely supported) probability distributions on
it. The Kleisli category for $D$ is the category $\textbf{Stoch}$ of
\emph{stochastic maps}; a map in the Kleisli category $f : X \to DY$ send each
$x$ in $X$ to a \emph{probability distribution} $f(x)$ on $Y$, which we could
interpret as the probability that $x$ is mapped to a given element of $Y$.

A Markov system is a stochastic map $u : S \to DS$ giving for each state $s$ a
probability distribution $u(s)$ on states. We interpret the probability
$u(s)(t)$ as the likelihood that state $s$ will transition to state $t$. Often
these numbers $u(s)(t)$ are arranged in a matrix and one calculates the
interated transition probabilities $u^{\circ n}$ by multiplying this matrix.
This can be seen formally from the (strong) monad morphism $i : D \to \Rb \otimes -$
from $D$ to the $\Rb$-linear monad $\Rb \otimes -$ that sends a set $X$ to the
free $\Rb$-vector space $\Rb \otimes X$ generated by it. This monad morphism
interprets every probability distribution $p$ on a set $X$ as a formal $\Rb$-linear
combination of elements of $X$.


The BiKleisli category $\textbf{BiKleisli}(C \times -, D)$ therefore consists of
\emph{contextualized stochastic maps}, or functions $f : C \times X \to D Y$
consisting of a stochastic map $f_c : X \to DY$ for every context $c \in C$.
\begin{defn}
An \emph{open Markov system} is a $\textbf{BiKleisli}(((-) \times (-),
D))$-dynamical system. An open Markov system consists of:
\begin{itemize}
\item A set $S$ of states, a set $O$ of outputs, and a set $I$ of inputs, and
\item A readout function $r : S \to O$, and an stochastic update function $S
  \times I \to D S$.
\end{itemize}
\end{defn}

We can see a morphism of open Markov systems as a \emph{hierarchical planner}.
Suppose we have a function $\alpha  : S \times A \to D S$ that takes a state $s \in S$
and an action $a \in A$ yields a distribution of expected new states $\alpha(s,
a) \in DS$. Then we have an
open markov system $\lens{\alpha}{\id} : \lens{S}{S} \to \lens{A}{S}$. Given
another such system $\lens{\alpha'}{\id} : \lens{S'}{S'} \to \lens{A'}{S'}$
which we will think of as operating at a higher scale, a
map of open Markov systems $(g, \lens{p}{g}) : \lens{\alpha}{\id} \to
\lens{\alpha'}{\id}$ consists of a lens $\lens{p}{g} : \lens{A}{S} \to
\lens{A'}{S'}$ (observational equivalence forces that the map $S \to S'$ is
$g$). We think of this as consisting of:
\begin{itemize}
\item A coarse graining $g : S \to S'$ saying how each low level state
  corresponds to a high level state.
  \item A planner $p : S \times A' \to D A$ saying which low level action one
    expects to take given a low level state and a high level action.  
  \end{itemize}
  Where the following diagram must commute:

  \[
      \begin{tikzcd}
S \times A' \arrow[r, "\Delta \times A'"] \arrow[d, "g \times A'"'] & S^3 \times A' \arrow[r, "S^2 \times p"]  & S^2 \times DA \arrow[d, "S \times \sigma"]      \\
S' \times A' \arrow[d, "u'"']                                       &                                          & S \times D(S \times A) \arrow[d, "S \times Du"] \\
D S'                                                                & S \times D S \arrow[l, "Dg \circ \pi_1"] & S\times DD(S) \arrow[l, "S \times \mu"]        
\end{tikzcd}
  \]

We interpret this as saying that if one begins with a low level state and high
level action and seeks the resulting high level state that one gets from acting
on the (coarse graining of) that state and action, it suffices to plan and execute a low level action and then coarse grain the
resulting low level state.

\subsection{$M$-dynamical systems as Automata via Coalgebras for Endofunctors}

Suppose that $\Ca$ is a cartesian closed category and that $M$ is a strong monad
on it. Then the data
$$\lens{u}{r} : \lens{S}{S} \to \lens{I}{O}$$
of an $M$-dynamical system in $\Ca$ can be summarized in the single morphism
$$(r, \hat{u}) : S \to O \times MS^I$$
where $\hat{u} : S \to MS^I$ is the transpose of $u : S \times I \to MS$. This
is an example of a \emph{coalgebra} for the endofunctor $X \mapsto O \times
MX^I$ of $\Ca$.

We recall the definition of coalgebras for endofunctors now.
\begin{defn}
Let $F : \Ca \to \Ca$ be an endofunctor of a category $\Ca$. A \emph{coalgebra}
for $F$ is a morphism $\alpha : S \to FS$. A \emph{homomorphism} $f : \alpha \to
\beta$ of coalgebras is a map $f : S \to S'$ so that the following square
commutes:
\[
\begin{tikzcd}
S \arrow[d,"f"'] \arrow[r,"\alpha" ] & FS \arrow[d,"Ff" ] \\
S' \arrow[r,"\beta"'] & FS'
\end{tikzcd}
\]

We denote the category of coalgebras of $F$ by $\textbf{Coalg}_F$. Given a
natural transformation $\phi : F \to G$, we get a functor $\textbf{Coalg}_\phi :
\textbf{Coalg}_{F} \to \textbf{Coalg}_G$ sending $S \xto{\alpha} FS$ to the
composite $S \xto{\alpha} FS \xto{\phi_S} GS$. This gives a functor
$$\textbf{Coalg} : \textbf{End}(\Ca) \to \Cat.$$
\end{defn}

For a strong monad $M$, a coalgebra for the endofunctor $X \mapsto O \times
MX^I$ is a \emph{$M$-automata}. For example, if $\Ca$ is the category of sets
and $O = \{\text{accept}, \text{reject}\}$, then an $\id$-automata is a function
$$\alpha : S \to \{\text{accept}, \text{reject}\} \times S^I$$
consisting of a function $\alpha_1 : S \to \{\text{accept}, \text{reject}\}$
marking each state as either an accept state or a reject state, and a function
$\alpha_2 : S \to S^I$ sending each state to its transition function that aways
an input $i$ from the input alphabet $I$ and changes state. This is the
classical notion of \emph{deterministic automaton}. If we let $M$ be the
powerset monad, we get $\emph{non-deterministic}$ automata, and if $M$ is the
monad of probability distributions we get $\emph{probabalistic}$ automata.


In this section, we will show that there is a functor
$$\text{endo} : \textbf{Lens}_{M} \to \textbf{End}(\Ca)$$
given by 
$$\text{endo}\lens{I}{O} := X \mapsto O \times MX^I.$$
We will also show that 
$$\textbf{Dyn}_{(-)} \simeq \textbf{Coalg} \circ \text{endo}$$
the category of $\lens{I}{O}$-$M$-dynamical systems is naturally equivalent to the
category of coalgebras for the endofunctor $X \mapsto O \times MX^I$. This gives
us a presentation of the category of $M$-dynamical systems as $M$-automata.

We begin by defining the action of the functor $\text{endo}$ on $M$-lenses.
Given $\lens{f^{\sharp}}{f} : \lens{I}{O} \to \lens{I'}{O'}$, we will get a
natural transformation $\text{endo}\lens{f^{\sharp}}{f}_X : O \times MX^I \to O'
\times MX^{I'}$ consisting of two components:
\begin{itemize}
\item The first component $O \times MX^I \to O'$ sends $(o, u)$ to $f(o)$.
$$O \times MX^I \xto{\pi_O} O \xto{f} O'.$$
\item The second component $O \times MX^I \to MX^{I'}$ send $(o, u)$ to the
  function $j \mapsto u(f^{\sharp}(o, j))$. This is the transpose of the
  following composite:
$$O \times I' \times MX^I \xto{f^{\sharp} \times MX^I} I \times MX^I
\xto{\text{ev}} MX$$
where $\text{ev} : I \times MX^I \to MX$ is the evalutation function, transpose
to the identity $MX^I \to MX^I$.
\end{itemize}

\begin{prop}
The assignment $\text{endo}: \textbf{Lens}_M \to \textbf{End}(\Ca)$ is a functor.
\end{prop}
\begin{proof}
Identity lenses are clearly sent to the identity natural transformation. Now,
suppose we have lenses
\end{proof}




\subsection{Example: Continuous Dynamical Systems}

Consider the indexed category of submersions $\textbf{Mfd}
\downarrow_{\textbf{Subm}} - : \textbf{Mfd}\op \to \Cat$, equipped with the
section $\epsilon M := TM \to M$ assigning each manifold its tangent bundle.

\begin{defn}
A \emph{continuous time dynamical system} is a $(\textbf{Mfd}
\downarrow_{\textbf{Subm}} -, T)$-dynamical system. That is, a continuous time
dynamical system consists of:
\begin{itemize}
\item A manifold $S$ of \emph{states}.
\item A manifold $O$ of \emph{outputs}.
\item A submersion $i : I \to O$ which we think of as assigning each output $o \in
  O$ to the space $I_o := i\inv(o)$ of inputs the system can receive given the
  output $o$.
\item A $(\textbf{Mfd}
\downarrow_{\textbf{Subm}} -)$-lens $\lens{u}{r} : \lens{TS}{S} \to \lens{I}{O}$
consisting of:
\begin{itemize}
\item a smooth \emph{readout} function $r : S \to O$, and
  \item a smooth \emph{update} function $u : S \otimes_O I \to TS$ over $S$,
    sending a state $s$ and an input $i \in I_{r(s)}$ to a vector $u(s, i)$
    tangent to $s$ saying where the system should transition to next. 
\end{itemize}
\end{itemize}

\end{defn}














