\usepackage{fullpage}
\usepackage{fancyhdr}
\usepackage{amsmath,amssymb,amsthm}
\usepackage{tikz}
\usepackage{wrapfig} % For inserting graphics
\usepackage{graphicx} % For inserting graphics
\usepackage{tikz-cd}
\usepackage[margin=1in]{geometry} % To shape the margins
\usepackage[backend=biber]{biblatex}
\usepackage{tipa} % For \textesh, the shape modality
\usepackage{mathtools} % for pre-superscripts
\usepackage{caption}
\usepackage{thmtools, thm-restate} % for forward references

\usetikzlibrary{decorations.pathreplacing} % for drawing coverings.

\newlength{\perspective} %% For drawing cubes

\graphicspath{graphics/}

%%% Mathcal
\DeclareMathOperator{\Aa}{\mathcal{A}}
\DeclareMathOperator{\Ba}{\mathcal{B}}
\DeclareMathOperator{\Ca}{\mathcal{C}}
\DeclareMathOperator{\Da}{\mathcal{D}}
\DeclareMathOperator{\Ea}{\mathcal{E}}
\DeclareMathOperator{\Fa}{\mathcal{F}}
\DeclareMathOperator{\Ga}{\mathcal{G}}
\DeclareMathOperator{\Ha}{\mathcal{H}}
\DeclareMathOperator{\Ka}{\mathcal{K}}
\DeclareMathOperator{\La}{\mathcal{L}}
\DeclareMathOperator{\Ma}{\mathcal{M}}
\DeclareMathOperator{\Na}{\mathcal{N}}
\DeclareMathOperator{\Oa}{\mathcal{O}}
\DeclareMathOperator{\Pa}{\mathcal{P}}
\DeclareMathOperator{\Qa}{\mathcal{Q}}
\DeclareMathOperator{\Ra}{\mathcal{R}}
\DeclareMathOperator{\Sa}{\mathcal{S}}
\DeclareMathOperator{\Ta}{\mathcal{T}}
\DeclareMathOperator{\Ua}{\mathcal{U}}
\DeclareMathOperator{\Va}{\mathcal{V}}
\DeclareMathOperator{\Wa}{\mathcal{W}}
\DeclareMathOperator{\Xa}{\mathcal{X}}
\DeclareMathOperator{\Ya}{\mathcal{Y}}
\DeclareMathOperator{\Za}{\mathcal{Z}}

%%% Mathbb
\DeclareMathOperator{\Ab}{\mathbb{A}}
\DeclareMathOperator{\Bb}{\mathbb{B}}
\DeclareMathOperator{\Cb}{\mathbb{C}}
\DeclareMathOperator{\Db}{\mathbb{D}}
\DeclareMathOperator{\Eb}{\mathbb{E}}
\DeclareMathOperator{\Fb}{\mathbb{F}}
\DeclareMathOperator{\Gb}{\mathbb{G}}
\DeclareMathOperator{\Hb}{\mathbb{H}}
\DeclareMathOperator{\Kb}{\mathbb{K}}
\DeclareMathOperator{\Lb}{\mathbb{L}}
\DeclareMathOperator{\Mb}{\mathbb{M}}
\DeclareMathOperator{\Nb}{\mathbb{N}}
\DeclareMathOperator{\Ob}{\mathbb{O}}
\DeclareMathOperator{\Pb}{\mathbb{P}}
\DeclareMathOperator{\Qb}{\mathbb{Q}}
\DeclareMathOperator{\Rb}{\mathbb{R}}
\DeclareMathOperator{\Sb}{\mathbb{S}}
\DeclareMathOperator{\Tb}{\mathbb{T}}
\DeclareMathOperator{\Ub}{\mathbb{U}}
\DeclareMathOperator{\Vb}{\mathbb{V}}
\DeclareMathOperator{\Wb}{\mathbb{W}}
\DeclareMathOperator{\Xb}{\mathbb{X}}
\DeclareMathOperator{\Yb}{\mathbb{Y}}
\DeclareMathOperator{\Zb}{\mathbb{Z}}

\DeclareMathOperator{\pc}{\mathfrak{p}}
\DeclareMathOperator{\Ic}{\mathfrak{I}}

%%% Operators -- Algebra
\DeclareMathOperator{\im}{\mathsf{im}}
\DeclareMathOperator{\Hom}{\mathsf{Hom}}
\DeclareMathOperator{\Spec}{\mathsf{Spec}}
\DeclareMathOperator{\Gr}{\mathsf{Gr}}
\DeclareMathOperator{\join}{\vee}
\DeclareMathOperator{\meet}{\wedge}

%%% Operators -- Categorical
\DeclareMathOperator{\op}{^\text{op}}
\DeclareMathOperator{\inv}{^{-1}}
\DeclareMathOperator{\id}{\mathsf{id}}
\renewcommand{\ker}{\mathsf{ker}}
\DeclareMathOperator{\coker}{\mathsf{coker}}
\DeclareMathOperator{\Aut}{\mathsf{Aut}}
\DeclareMathOperator{\BAut}{\mathsf{BAut}}
\DeclareMathOperator{\Map}{\mathsf{Map}}
\DeclareMathOperator{\Gal}{\mathsf{Gal}}
\DeclareMathOperator{\colim}{\mathsf{colim}}
\DeclareMathOperator{\End}{\mathsf{End}}

%%% Operators -- Type Theory
\DeclareMathOperator{\ev}{\textbf{ev}}
\DeclareMathOperator{\Prop}{\textbf{Prop}}
\DeclareMathOperator{\Type}{\textbf{Type}}
\DeclareMathOperator{\precomp}{^{\ast}}
\DeclareMathOperator{\postcomp}{_{\ast}}
\DeclareMathOperator{\refl}{\mathsf{refl}}
\DeclareMathOperator{\fib}{\mathsf{fib}}
\DeclareMathOperator{\cofib}{\type{cofib}}
\DeclareMathOperator{\Zero}{\textbf{0}}
\DeclareMathOperator{\One}{\textbf{1}}
\DeclareMathOperator{\fst}{\term{fst}}
\DeclareMathOperator{\snd}{\term{snd}}
\DeclareMathOperator{\base}{\term{base}}
\DeclareMathOperator{\funext}{\term{funext}}
\DeclareMathOperator{\tr}{\term{tr}}
\DeclareMathOperator{\ap}{\term{ap}}
\DeclareMathOperator{\at}{\term{\, at\,}}
\DeclareMathOperator{\glue}{\term{glue}}
\DeclareMathOperator{\inl}{\term{inl}}
\DeclareMathOperator{\inr}{\term{inr}}
\DeclareMathOperator{\esh}{\text{\textesh}}
\let\bang\relax % aaahhhhh
\DeclareMathOperator{\bang}{!}
\DeclareMathOperator{\defin}{\term{-def}}
\DeclareMathOperator{\Bool}{\textbf{Bool}}
\newcommand{\sslash}{\mathbin{/\mkern-6mu/}} % double slash for homotopy quotients

%%% Categories
\DeclareMathOperator{\Set}{\textbf{Set}}
\DeclareMathOperator{\Top}{\textbf{Top}}
\DeclareMathOperator{\Grph}{\textbf{Grph}}
\DeclareMathOperator{\AbCat}{\textbf{Ab}}
\DeclareMathOperator{\Cat}{\textbf{Cat}}
\DeclareMathOperator{\Vect}{\textbf{Vect}}
\DeclareMathOperator{\Ring}{\textbf{Ring}}
\DeclareMathOperator{\Mod}{\textbf{Mod}}
\DeclareMathOperator{\Alg}{\textbf{Alg}}
\DeclareMathOperator{\Aff}{\textbf{Aff}}
\DeclareMathOperator{\SqFree}{\textbf{SqFree}}
\DeclareMathOperator{\SQC}{\Alg_{\textbf{SQC}}}


%%% Commands -- Category Theory
\newcommand{\wlim}[1]{\lim\nolimits_{#1}\!}
\newcommand{\wcolim}[1]{\text{colim}_{#1}}

%Yoneda
\font\maljapanese=dmjhira at 2ex % you can change this "2ex" value
\def\yo{\textrm{\maljapanese\char"48}\!}

%%% Commands -- Type Theory
\newcommand{\lam}[1]{\lambda {#1}.\,}       % \lam{x : X} Y
\newcommand{\dprod}[1]{({#1}) \to }           %  \dprod{x : X} Y(x)
\newcommand{\dsum}[1]{({#1}) \times }       % \dsum{x : X} Y(x)
\newcommand{\type}[1]{\mathsf{{#1}}}
\newcommand{\term}[1]{\mathsf{{#1}}}
\newcommand{\ax}[1]{\mathsf{{#1}}}
\newcommand{\trunc}[1]{\left\lVert#1\right\rVert}       % truncation
\newcommand{\ph}{\vphantom{)}} % For iterated upper and lower modalities
\newcommand{\BB}{\term{B}}
\newcommand{\pt}{\term{pt}}

%%% Commands -- Autobots
\newcommand{\prism}{\mathsf{Prism}}
\newcommand{\ob}{\mathsf{ob}}
\newcommand{\Finset}{\mathsf{Finset}}
\newcommand{\TFS}{\mathsf{TFS}}
\newcommand{\Lens}{\mathsf{Lens}}
\newcommand{\LLens}{\mathsf{LawLens}}
\newcommand{\SLens}{\mathsf{SimpLens}}
\newcommand{\Dyn}{\mathsf{Dyn}}
\newcommand{\lens}[2]{\begin{pmatrix}{#1} \\ {#2} \end{pmatrix}}



%%% Commands -- Arrows and Equals
\newcommand{\xto}[1]{\xrightarrow{#1}}
\newcommand{\todot}{\xrightarrow{\bullet}}
\newcommand{\pto}{\,\cdot\kern-.1em{\to}\,}

% xmapsto
\makeatletter
\providecommand*{\xmapstofill@}{%
  \arrowfill@{\mapstochar\relbar}\relbar\rightarrow
}
\providecommand*{\xmapsto}[2][]{%
  \ext@arrow 0395\xmapstofill@{#1}{#2}%
}

% xequals
\newcommand{\xequals}[1]{\overset{{#1}}{=\joinrel=}}
\newcommand{\equalsdot}{\xequals{\bullet}}

% Proarrows
\makeatletter
\def\slashedarrowfill@#1#2#3#4#5{%
  $\m@th\thickmuskip0mu\medmuskip\thickmuskip\thinmuskip\thickmuskip
   \relax#5#1\mkern-7mu%
   \cleaders\hbox{$#5\mkern-2mu#2\mkern-2mu$}\hfill
   \mathclap{#3}\mathclap{#2}%
   \cleaders\hbox{$#5\mkern-2mu#2\mkern-2mu$}\hfill
   \mkern-7mu#4$%
}
\def\rightslashedarrowfill@{%
  \slashedarrowfill@\relbar\relbar\mapstochar\rightarrow}
\newcommand\xslashedrightarrow[2][]{%
  \ext@arrow 0055{\rightslashedarrowfill@}{#1}{#2}}
\makeatother

\newcommand{\xtopro}[1]{\xslashedrightarrow{#1}} % Proarrows
\newcommand{\topro}{\xslashedrightarrow{}}

\tikzset{
    vert/.style={anchor=south, rotate=90, inner sep=.5mm}
} % For vertical \sim in tikzcd, use "\sim" vert

%% Corner quotes
\newcommand{\internal}[1]{\ulcorner{#1}\urcorner}


\newtheorem{thm}{Theorem}[section]

%%% Theorem Environments
\theoremstyle{definition}
\newtheorem{defn}[thm]{Definition}
\newtheorem{tentative}[thm]{Tentative Definition}
\newtheorem{principle}{Principle}
\newtheorem{question}{Question}
\newtheorem{ex}{Example}
\newtheorem{axiom}{Axiom}
\newtheorem{exercise}{Exercise}[section]
\newtheorem{rmk}[thm]{Remark}
\newtheorem*{solution}{Solution}
\newtheorem*{acknowledgements}{Acknowledgements}


\newtheorem{lem}[thm]{Lemma}
\newtheorem{notation}{Notation}
\newtheorem*{claim}{Claim}
\newtheorem*{fclaim}{False Claim}
\newtheorem{obs}[thm]{Observation}
\newtheorem{conjecture}[thm]{Conjecture}
\newtheorem{cor}[thm]{Corollary}
\newtheorem{prop}[thm]{Proposition}
\newtheorem{con}[thm]{Construction}